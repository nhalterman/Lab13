\documentclass[12pt]{scrartcl}


\setlength{\parindent}{0pt}
\setlength{\parskip}{.25cm}

\usepackage{graphicx}

\usepackage{xcolor}

\definecolor{darkred}{rgb}{0.5,0,0}
\definecolor{darkgreen}{rgb}{0,0.5,0}
\usepackage{hyperref}
\hypersetup{
  letterpaper,
  colorlinks,
  linkcolor=red,
  citecolor=darkgreen,
  menucolor=darkred,
  urlcolor=blue,
  pdfpagemode=none,
  pdftitle={CSCE 156 Lab Handout},
  pdfauthor={Christopher M. Bourke},
  pdfsubject={},
  pdfkeywords={}
}

\definecolor{MyDarkBlue}{rgb}{0,0.08,0.45}
\definecolor{MyDarkRed}{rgb}{0.45,0.08,0}
\definecolor{MyDarkGreen}{rgb}{0.08,0.45,0.08}

\definecolor{mintedBackground}{rgb}{0.95,0.95,0.95}
\definecolor{mintedInlineBackground}{rgb}{.90,.90,1}

%\usepackage{newfloat}
\usepackage[newfloat=true]{minted}
\setminted{mathescape,
               linenos,
               autogobble,
               frame=none,
               framesep=2mm,
               framerule=0.4pt,
               %label=foo,
               xleftmargin=2em,
               xrightmargin=0em,
               startinline=true,  %PHP only, allow it to omit the PHP Tags *** with this option, variables using dollar sign in comments are treated as latex math
               numbersep=10pt, %gap between line numbers and start of line
               style=default, %syntax highlighting style, default is "default"
               			    %gallery: http://help.farbox.com/pygments.html
			    	    %list available: pygmentize -L styles
               bgcolor=mintedBackground} %prevents breaking across pages
               
\setmintedinline{bgcolor={mintedBackground}}
\setminted[text]{bgcolor={mintedBackground},linenos=false,autogobble,xleftmargin=1em}
%\setminted[php]{bgcolor=mintedBackgroundPHP} %startinline=True}
\SetupFloatingEnvironment{listing}{name=Code Sample}
\SetupFloatingEnvironment{listing}{listname=List of Code Samples}


\title{CSCE 156 -- Computer Science II}
\subtitle{Lab 13.0 - Sorting}
\author{~}
\date{~}

\begin{document}

\maketitle

\section*{Prior to Lab}

\begin{enumerate}
  \item Review this laboratory handout prior to lab.
  \item Review insertion sort and quick sort from the class 
  	notes or from the following resources:
	\begin{itemize}
	  \item \url{http://en.wikipedia.org/wiki/Insertion_sort}
	  \item \url{http://en.wikipedia.org/wiki/Quick_sort}
	\end{itemize}
  \item Familiarize yourself with sorting algorithms using the
  	following site: \url{http://www.sorting-algorithms.com/}
\end{enumerate}

\section*{Lab Objectives \& Topics}
Following the lab, you should be able to:
\begin{itemize}
  \item Sort a list of objects by using the 
    \mintinline{java}{Comparable<T>} interface and various sorting 
    methods including the \mintinline{java}{Collections.sort()} 
    method.
  \item Be familiar with Insertion Sort and Quick Sort algorithms 
    and be able to adapt them to sort comparable objects.
  \item Compare and contrast the performance of various sorting methods
  \item Empirically measure the relative performance of algorithms 
    with respect to the input size. 
\end{itemize}

\section*{Peer Programming Pair-Up}

To encourage collaboration and a team environment, labs will be
structured in a \emph{pair programming} setup.  At the start of
each lab, you will be randomly paired up with another student 
(conflicts such as absences will be dealt with by the lab instructor).
One of you will be designated the \emph{driver} and the other
the \emph{navigator}.  

The navigator will be responsible for reading the instructions and
telling the driver what to do next.  The driver will be in charge of the
keyboard and workstation.  Both driver and navigator are responsible
for suggesting fixes and solutions together.  Neither the navigator
nor the driver is ``in charge.''  Beyond your immediate pairing, you
are encouraged to help and interact and with other pairs in the lab.

Each week you should alternate: if you were a driver last week, 
be a navigator next, etc.  Resolve any issues (you were both drivers
last week) within your pair.  Ask the lab instructor to resolve issues
only when you cannot come to a consensus.  

Because of the peer programming setup of labs, it is absolutely 
essential that you complete any pre-lab activities and familiarize
yourself with the handouts prior to coming to lab.  Failure to do
so will negatively impact your ability to collaborate and work with 
others which may mean that you will not be able to complete the
lab.  

\section*{Sorting}

Being able to organize and retrieve information in large datasets is 
a big research area with numerous applications.  At the heart of any 
data mining endeavor is the fundamental operation of sorting.

For this lab, we have provided a file containing United States 
geographical data (specifically, a comprehensive list of 80264 
geographical locations identified by 5-digit zip code, city, state 
and latitude/longitude).  Functionality has already been provided 
for processing this data file and creating an array of 
\mintinline{java}{Location} objects to hold this data.  Each time you 
run the \mintinline{java}{main()} method in the 
\mintinline{java}{SortingPerformance} class, a random selection of 
locations is loaded.  The parameter, \mintinline{java}{n} limits the 
number of locations loaded from the file so that you'll be able to 
easily run experiments on different input sizes.

It will be your task to determine how this geographical data should 
be sorted, develop several sorting algorithms to actually sort the 
data, and empirically evaluate the running time of each method.

\subsection*{Comparable Interface}

Numerical and string data types have a natural ordering that is 
understood by a computer language.  However, user defined types 
such as the \mintinline{java}{Location} objects do not have an 
obvious ordering; it may be possible to order locations by an 
individual field (zip code, state, city) or a combination of those 
fields.  Java provides a means for you to define exactly how 
instances of your class should be ordered by implementing the 
\mintinline{java}{Comparable<T>} interface.

If a class implements the \mintinline{java}{Comparable<T>} interface, 
it must provide an implementation for the following method.

\mintinline{java}{public int compareTo(T item)}

The general contract of this method is as follows:
\begin{itemize}
  \item It returns a negative number if this object should precede 
    the item object
  \item It returns zero if this object is ``equivalent'' to the 
    item object
  \item It returns a positive number if this object should come 
    after the item object
\end{itemize}
    
\subsection*{Sorting Algorithms}
    
You will be comparing four different sorting algorithms: selection 
sort, insertion sort, quick sort, and the sorting method provided by 
the JDK (Java Developer's Kit, specifically the 
\mintinline{java}{Arrays.sort(Object[])} method).  The selection sort 
algorithm and the method to call the JDK's sorting algorithm have 
already been provided for you in the SortingAlgorithms class.  Refer 
to the selection sort method especially if you are still unclear on 
how to use the \mintinline{java}{compareTo()} method.  You will need 
to adapt the insertion sort and quick sort algorithms from your text 
to sort \mintinline{java}{Location} objects in the given array.  Be 
sure to sort and return a copy of the array, not the array itself.

\subsection*{Comparing Running Times}

To compare how each sorting algorithm performs, you will setup an 
experiment and calculate how much time it actually takes to sort the 
array of locations by using the \mintinline{java}{System.nanoTime()} 
method.  This method returns the current system time, more-or-less 
precise to a nanosecond (1 second = 1,000 milliseconds = 1,000,000 
microseconds = 1,000,000,000 nanoseconds).  By taking a snapshot of 
the system time before and after a method call, you can compute 
(roughly) the total elapsed time.  As an example, consider the 
following code snippet.

\begin{minted}{java}
long start = System.nanoTime();
Location sorted[] = SortingAlgorithms.insertionSort(sp.getLocations());
long end = System.nanoTime();
long elapsedTimeNs = (end - start);
\end{minted}

\section*{Activities}

Clone the starter code for this lab from GitHub using the following
url: \url{https://github.com/cbourke/CSCE156-Lab13}.

\begin{enumerate}
  \item Implement the \mintinline{java}{compareTo()} method in the 
    \mintinline{java}{Location} class.  You may use whatever criteria
    you wish (city/state or latitude/longitude etc.).
  \item Implement the \mintinline{java}{insertionSort()} and 
    \mintinline{java}{quickSortRecursive()} methods in the 
    \mintinline{java}{SortingAlgorithms} class.  Do this by adapting 
    the algorithms as presented in your text book or other
    resource (algorithms are also available in our
    \href{https://bitbucket.org/chrisbourke/computersciencei/raw/44fb9b39be3221dc02c1b5d0712f9b9f03260e46/ComputerScienceOne.pdf}{CS1 textbook}.
  \item Debug your code and verify that each sorting algorithm is 
    correctly sorting by using the \mintinline{java}{printLocationArray()} 
    method provided to you to output the array contents. 
  \item Perform several timed experimental runs of each of your 
    algorithms on various array sizes as outlined in the worksheet
\end{enumerate}


\section*{Submission}

We have included a test suite of unit tests written in JUnit 
(\url{https://junit.org/junit5/}) a popular unit testing framework for
Java.  Even though the test driver (in the \mintinline{text}{src/test}
source folder) has no main method, you can still run it in Eclipse and
get a report on how many of the tests passed, failed or resulted in 
an unexpected exception.  Be sure all of the unit tests pass before
submitting your source files through webhandin.  You can rerun this
test suite in the webgrader to ensure everything works.  

\subsection*{Advanced Activities (Optional)}

\begin{enumerate}
  \item Examine the experimental data for each of the sorting algorithms 
    on the various input sizes.  Under the assumption that we have 1 
    million records, make a prediction, based on the observed running 
    times on how long each algorithm would take to execute.  Find a (or 
    generate a fake) fake dataset of 1 million entries and run your 
    experiment.  Do your predictions match the actual running time?
  \item Implementing the \mintinline{java}{Comparable} interface means 
    that \mintinline{java}{Location} objects can only be ``naturally 
    ordered'' in one way.  It is often more flexible to instead use 
    \mintinline{java}{Comparator} class to enable a user to order objects 
    in any order that they want while reusing your methods.  Rewrite 
    each of the sorting methods to instead use a 
    \mintinline{java}{Comparator<Location>} instance passed in to the 
    method. 
\end{enumerate}

\end{document}
