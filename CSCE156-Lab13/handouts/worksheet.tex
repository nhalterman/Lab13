\documentclass[12pt]{scrartcl}

\usepackage{fullpage}

\setlength{\parindent}{0pt}
\setlength{\parskip}{.25cm}

\usepackage{graphicx}

\usepackage{xcolor}

\definecolor{darkred}{rgb}{0.5,0,0}
\definecolor{darkgreen}{rgb}{0,0.5,0}
\usepackage{hyperref}
\hypersetup{
  letterpaper,
  colorlinks,
  linkcolor=red,
  citecolor=darkgreen,
  menucolor=darkred,
  urlcolor=blue,
  pdfpagemode=none,
  pdftitle={CS2 - Lab Worksheet},
  pdfkeywords={}
}

\definecolor{MyDarkBlue}{rgb}{0,0.08,0.45}
\definecolor{MyDarkRed}{rgb}{0.45,0.08,0}
\definecolor{MyDarkGreen}{rgb}{0.08,0.45,0.08}

\definecolor{mintedBackground}{rgb}{0.95,0.95,0.95}
\definecolor{mintedInlineBackground}{rgb}{.90,.90,1}

%\usepackage{newfloat}
\usepackage[newfloat=true]{minted}
\setminted{mathescape,
               linenos,
               autogobble,
               frame=none,
               framesep=2mm,
               framerule=0.4pt,
               %label=foo,
               xleftmargin=2em,
               xrightmargin=0em,
               startinline=true,  %PHP only, allow it to omit the PHP Tags *** with this option, variables using dollar sign in comments are treated as latex math
               numbersep=10pt, %gap between line numbers and start of line
               style=default, %syntax highlighting style, default is "default"
               			    %gallery: http://help.farbox.com/pygments.html
			    	    %list available: pygmentize -L styles
               bgcolor=mintedBackground} %prevents breaking across pages
               
\setmintedinline{bgcolor={mintedBackground}}
\setminted[text]{bgcolor={mintedBackground},linenos=false,autogobble,xleftmargin=1em}
%\setminted[php]{bgcolor=mintedBackgroundPHP} %startinline=True}
\SetupFloatingEnvironment{listing}{name=Code Sample}
\SetupFloatingEnvironment{listing}{listname=List of Code Samples}

\begin{document}

\section*{CSCE 156 - Lab 13.0 - Sorting - Worksheet}

Names: \underline{\hspace{10cm}}

\begin{enumerate}
  \item Verify that your sorting algorithms are correctly sorting 
  by printing the content of the arrays.  Be prepared to demonstrate
  this to a lab instructor.
  
  \item Run some timed experiments as outlined in the lab handout 
  for each algorithm for various input sizes.  Note that you can 
  restrict the number of locations loaded from the data file by 
  changing the value of \mintinline{java}{n} in the \mintinline{java}{main}
  method of the \mintinline{java}{SortingPerformance} class.  
  Fill in the table below (for best results, run the experiment 
  at least three times each and take an average running time 
  unless you're feeling lazy).

\begin{table}[h]
\centering
\begin{tabular}{|l|l|l|l|l|l|}
\hline
Algorithm      & Theoretical Efficiency & \multicolumn{4}{l|}{Observed Performance (sec)}        \\ \hline
               &                        & $n = 2,000$ & $n = 4,000$ & $n = 8,000$ & $n = 16,000$ \\ \hline
Java Sort      & $O(n\log{(n)})$        &             &             &             &              \\[5ex] \hline
Selection Sort & $O(n^2)$               &             &             &             &              \\[5ex] \hline
Insertion Sort & $O(n^2)$               &             &             &             &              \\[5ex] \hline
Quick Sort     & $O(n\log{(n)})$        &             &             &             &              \\[5ex] \hline
\end{tabular}
\end{table}

 \item Without actually running the simulation, predict the running 
 time of each algorithm for $n = 64,000$ based on the theoretical 
 efficiency and observed running time.
 
 \item According to your experiments, is there a clear ranking of 
 the sorting algorithms?  If so, list them from best to worst.  
 Present your results to the lab instructor.

\end{enumerate}

Lab Instructor Signature\underline{\hspace{7.5cm}}

\end{document}
